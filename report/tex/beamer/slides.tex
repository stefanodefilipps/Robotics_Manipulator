\documentclass[11pt]{beamer}
\usetheme{CambridgeUS}
\usepackage[utf8]{inputenc}
\usepackage[english]{babel}
\usepackage{amsmath}
\usepackage{amsfonts}
\usepackage{amssymb}
\usepackage{xcolor}
\author[Stefano, Marco]{Stefano De Filippis \& Marco Menchetti}
\title[ROB2 project]{Task priority assignment with collision avoidance}
%\setbeamercovered{transparent} 
%\setbeamertemplate{navigation symbols}{} 
\logo{\includegraphics[scale=.1]{./images/logo.jpeg}}
\institute[Sapienza]{Sapienza - University of Rome} 
\date{} 
\subject{Robotics II} 
\begin{document}

\begin{frame}
\titlepage
\end{frame}

%\begin{frame}
%\tableofcontents
%\end{frame}

\begin{frame}{Why priority?}
\begin{itemize}
\item Decomposition of problems in many tasks.
\item Most problems \textcolor{red}{can't} be solved by just one task.
\item Error is kept on the tasks that \textcolor{red}{can't} be executed	\textbf{EXACTLY} 
\item More natural and smoother behavior.
\end{itemize}
\end{frame}

\begin{frame}{Collision avoidance. How?}
\begin{columns}
\begin{column}{.5\textwidth}
\begin{block}{Control points}

\end{block}
\end{column}
\begin{column}{.5\textwidth}
Figure of the KUKA and its control points
\end{column}
\end{columns}
\end{frame}

\begin{frame}{Tasks}

\end{frame}

\begin{frame}{Tasks: Cartesian positioning}

\end{frame}

\begin{frame}{Tasks: Link orientation}

\end{frame}

\begin{frame}{Tasks: Collision avoidance control points}

\end{frame}

\begin{frame}{Code}

\end{frame}

\begin{frame}{Results}

\end{frame}

\end{document}